## Bird Flight Data

Jerald Dosch sent this on 2014-01-21.  It might be a good example of where data come from.

Here's a very interesting example of a "big data" study.  
http://www.nytimes.com/2014/01/21/science/bird-data-confirms-that-vs-help-save-energy.html?ref=science
http://www.nature.com/nature/journal/v505/n7483/full/nature12939.html
They put devices on 14 individual birds that flew as a flock from Austria to Italy.  The data capture sounds amazing:  "The birds wore custom-made data loggers that allowed the researchers to track flapping, speed and direction. Weighing less than an ounce, the devices included an accelerometer, a gyroscope, a magnetometer, a memory card, a battery, a microcontroller and a GPS unit “much better than on your iPhone,” Dr. Usherwood said. It is accurate to about one foot and refreshes five times per second — the resolution necessary to track the birds’ positions in relation to one another. "

Amazing!  Every 5 seconds they recorded flapping, speed and direction for each of the 14 birds on this long journey.


They only looked at 7 minutes worth of data (that should be 1176 data samplings at one sampling each 5 seconds x 14 birds) from the 600 mile flight.  

The researchers analyzed the birds’ positions over seven minutes of flight, and compared those observations with theoretical predictions generated by aerodynamic models. The upward-moving swirls of air, called tip vortices, are a byproduct of winged flight, said Kenny Breuer, a professor in the school of engineering and a professor of ecology and evolutionary biology at Brown University, who with David Willis and other colleagues, developed the predictions. As wings push air down to generate lift, other air rises to the right and left of the wings, forming the vortices. 

An analysis of 24,000 flaps showed that the ibises on average adjusted their position and wing phase to optimize the lift from the vortices, and readjusted their phasing when they changed positions within the V.

The flock location data are at Supplementary Data 1 (504 KB) http://www.nature.com/nature/journal/v505/n7483/extref/nature12939-s2.zip in an XML format.  This might be a good data scraping example, perhaps useful in calculus.
